\documentclass[]{report}
\usepackage{tikz}
\newcommand{\inputtikz}[2]{%  
	\scalebox{#1}{\input{#2}}  
}
\usepackage[english]{babel}
\usepackage[utf8x]{inputenc}
\usepackage{amsmath}
\usepackage{graphicx}
\usepackage[colorinlistoftodos]{todonotes}
\usepackage{listings}
\usepackage{color}
\usepackage{amsmath}
\usepackage{amsfonts}
\usepackage{mathtools}
\usepackage{graphicx}
\usepackage{caption}
\definecolor{dkgreen}{rgb}{0,0.6,0}
\definecolor{gray}{rgb}{0.5,0.5,0.5}
\definecolor{mauve}{rgb}{0.58,0,0.82}
%opening

\begin{document}
	
\begin{titlepage}
	
	\newcommand{\HRule}{\rule{\linewidth}{0.5mm}} 
	
	\center 
	
	\textsc{\LARGE Université de Technologie de Compiegne}\\[1.5cm]
	\textsc{\Large SY19}\\[0.5cm] 
	\textsc{\large Machine Learning}\\[0.5cm]
		
	\HRule \\[0.4cm]
	{ \huge \bfseries Second Assignment}\\[0.4cm] 
	\HRule \\[1.5cm]
		
	\begin{minipage}{0.4\textwidth}
		\begin{flushleft} \large
			Aladin \textsc{TALEB} 
		\end{flushleft}
	\end{minipage}
	~
	\begin{minipage}{0.4\textwidth}
		\begin{flushright} \large
			Zineb \textsc{SLAM} 
		\end{flushright}
	\end{minipage}\\[2cm]

	{\large \today}\\[2cm] 

	\includegraphics[width=40mm]{Figures/utc.jpg}\\ % 

	\vfill
	
\end{titlepage}

\lstset{frame=tb,
	language=R,
	aboveskip=3mm,
	belowskip=3mm,
	showstringspaces=false,
	framexleftmargin=5mm,
	columns= fixed,
	numbers = left,
	basicstyle={\small\ttfamily},	
	numberstyle=\tiny\color{gray},
	keywordstyle=\color{blue},
	commentstyle=\color{dkgreen},
	stringstyle=\color{mauve},
	breaklines=true,
	breakatwhitespace=true,
	tabsize=3
}

	

\begin{abstract}
	

\end{abstract}


\tableofcontents



\section{Context}
This exercice aims to build the best classifier to recognize facial expressions based on a normalized photo of a given subject. The model will take advantage of a database of about 200 photos to learn how to classify the six different expressions : happiness, surprise, sadness, disgust, anger and fear.

\section{Dataset Description}
The very first step of our method consists in taking a look at the raw dataset. The dataset comprises 216 black and white photos of size 60 by 70, thus each case is described through 4200 features. It turns out that a non-negligible part of those features are null, hence should be removed from the dataset.

The following code removes the null features from the dataset :
\begin{lstlisting}
X.clean = X[,colSums(X) != 0]
\end{lstlisting}
Each photo is now described through 3660 non-null features.



\end{document}
